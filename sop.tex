\documentclass[11pt]{article}

\usepackage{times}
\usepackage{xspace}
\usepackage[margin=1in]{geometry}

\newcommand\lila{\textsc{L\={\i}la}\xspace}
\newcommand\inst{INSTITUTE\xspace}

\begin{document}

I first took an interest in computational linguistics 
while learning to speak Tagalog. 
Through the process, I learned a deep appreciation for the language's rich morphology,
and the immense complexity of human language.
My enthusiasm for studying language only increased
as I began as a undergraduate researcher in NLP at Harvard
under the mentorship of Stuart Shieber and Yonatan Belinkov at Harvard,
and subsequently began as a pre-doctoral investigator at AI2 
advised by Peter Clark and Ashish Sabharwal.
These mentors, as well as my many amazing colleagues at both institutions,
have inspired me to pursue a PhD in natural language processing (NLP).

Though some subfields of NLP research 
have advanced by leaps and bounds in the past few years,
our understanding of these systems remains unsatisfactory.
While transformers may get impressive results on, e.g., question answering,
we do not know what the fundamental limits 
of what current architectures can compute or approximate,
or what these models are doing internally.
Do they learn syntactic rules? 
Can they generalize to new problems composed of known concepts?
In my PhD I hope to answer some of these questions
by \textbf{drawing on fields such as formal language theory and syntax
to develop practical and theoretical frameworks} 
for understanding modern NLP methods, 
especially when it comes to \textbf{understanding compositional generalization} 
and using \textbf{pre-trained language models as general-purpose systems}.

\paragraph{Understanding LMs through linguistics and formal language theory.}
Modern NLP has become largely divorced 
from our understanding of linguistics and cognition in humans.
I am interested in reconciling our understanding of both.
As my first foray into this effort, I set out to 
discover how modern LMs handle syntactic agreement.
In my first-author ACL 2021 paper and oral presentation~\cite{Finlayson2021CausalAO}
we intervene on individual neurons in large transformers 
to observe their causal effect~\cite{Pearl2001DirectAI} on syntactic agreement.
We find that transformers learn 
two distinct mechanisms for number agreement,
depending on whether the relevant tokens are adjacent or not.
Subsequent research has built on our work 
in order to analyze other linguistic phenomena 
such as distributivity~\cite{Ban2022TestingPL},
and the effect of relative clauses on agreement~\cite{Ravfogel2021CounterfactualIR}.
During my PhD I hope to continue this line of research 
by exploring how LMs handle (or fail to handle) 
other linguistic phenomena. 
I am particularly interested in linguistic generalization.
For instance, perhaps causal analysis could help us understand
how transformers handle and acquire new words inference time
by intervening on contextualized embeddings within the model.
Strengthening our understanding of how 
LMs deal with well-studied linguistic phenomena
can give insight into what inductive biases these models learn 
and how to make them better. 
I also believe that insights from computational methods 
can contribute to the field of linguistics
by empirically lower-bounding the computational ingredients 
for producing various phenomena that we see in human speech.

As noted earlier, I am also excited by projects 
that borrow from computation and formal language theory
to generate insights about LMs.
This approach makes it possible to answer 
questions that might otherwise be very difficult to approach.
For instance, I used regular languages
to measure the capabilities of 
\textbf{transformers as general instruction followers}
in RegSet, my first-author EMNLP 2022 paper 
and oral presentation~\cite{Finlayson2022WhatMI}. 
Large, pre-trained LMs can solve some NLP tasks 
by conditioning their generations on natural language instructions 
for the task~\cite{mishra2021crosstask, Wei2021FinetunedLM}. 
However, the complexity of natural language makes difficult to
predict what types of instructions may be beyond the grasp of Transformers.
To solve this predicament, 
we propose a controllable proxy for studying instruction learning
by studying LMs ability to learn interpret regular expressions
and recognize their strings.
Our experiments lead us to a number of intriguing hypotheses 
about what makes instruction learning hard, 
including evidence that even large transformers struggle with modular counting 
(e.g., determining whether something is even or odd). 
Subsequent work~\cite{Merrill2022LogPrecisionTA} 
has built on our formalization 
to obtain important theoretical results
by using circuit complexity theory
to formally characterize transformers ability to execute instructions in the form of a circuit.
In my PhD, I hope to continue developing both theoretical and empirical frameworks 
for understanding neural networks through formal languages and computation theory.
For instance, perhaps we can prove theoretical upper bounds 
for what transformers can learn from instructions,
or characterize the additional computational power 
afforded to transformers 
when they are allowed to generate multiple intermediate tokens 
before producing an output.
On the empirical side, we can use formal languages to develop benchmarks 
to isolate and measure progress towards specific abilities 
in transformers that we hope to see in natural language settings.
This approach takes advantage of the well studied properties of formal language
and gives us fine-grained control over the data.

\paragraph{Generalization}
My work on RegSet bridges to another area of research I hope to continue in during my PhD: generalization.
Research has shown that neural models consistently to struggle with
compositional generalization~\cite{Lake2018GeneralizationWS}. 
This bodes poorly for models in succeeding on tasks  
like the instruction following regime I studied with RegSet 
where the space of task descriptions is both intractably large 
and highly compositional.
I am interested in studying and developing models and methods
that tackle these types of issues.

Some of my past and current work deals with generalization 
by attempting to develop methods for utilizing pre-trained language models
to their full capacity as general problem solvers.
In a preprint currently under submission to ICLR 2023~\cite{Khot2022DecomposedPA},
I develop a modular prompting method for recursively prompting large LMs 
in order to vastly improve length generalization 
compared to few-shot and step-by-step reasoning style prompting.
Currently, I am developing methods for using language models 
to decode their own task-specific prompts.
One exciting application of this method would be to explore 
the possibility of adversarial prompts: 
prompts that appear to elicit one behavior 
but cause the model to exhibit another.
Research into adversarial prompts 
is an important step towards to developing 
secure and ethical general-purpose NLP systems. 

A crucial part of developing general-purpose systems is be able to evaluate them.
As an example, 
current evaluation schemes fail to holistically evaluate 
general-purpose math reasoning skills in LMs
because they are far too narrow in scope.
Towards improving evaluation in this area, 
I led a team of 11 researchers in compiling a
\textbf{comprehensive and diverse natural language math reasoning benchmark.} 
I introduce the benchmark, \lila~\cite{Mishra2022Lila}, 
in my first-author EMNLP 2022 paper
where we draw together a diverse set over 140K math problems.
We additionally provide valuable annotations for mathematical reasoning via program synthesis, 
and show that multi-task learning, 
combined with augmenting the model with a Python interpreter,
massively improves LMs ability to do math reasoning 
with explicit reasoning steps.
Our publicly released multitask model, christened Bh\=askara, 
outperforms similarly-sized T5 and GPT-Neo models
when fine-tuning on new mathematical reasoning tasks.
\lila shows that LMs 
in their current form 
are woefully deficient when it comes to math reasoning,
and highlights the need for these kinds of unified evaluations for 
aspiring general-purpose math reasoning models.
During my PhD I hope to continue to develop thoughtful and comprehensive evaluations 
to measure and promote research into models with greater general utility.

\paragraph{Future plans}
After my PhD and postdoc I hope to become a PI 
at an institution where I can achieve autonomy in choosing my research directions
while also pursuing my passion for mentorship teaching.
Ideally this means a professorship at a research university.
I value autonomy in research because I work best 
when I can focus my energy on projects 
that are intellectually interesting to me.
I have also especially become aware of the importance of mentorship 
throughout my undergrad and time at AI2.
I am passionate about promoting access to mentorship 
as a way to level the playing field 
for the next generation of researchers.

\paragraph{Fit for \inst} I am specifically interested in joining the CS program at 
\inst because \ldots Professor NAME's work on \ldots is particularly interesting to me given my interest in \ldots

\bibliography{ref}
\bibliographystyle{unsrt}

\end{document}

