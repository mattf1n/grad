\documentclass[11pt]{article}

\usepackage[margin=1in]{geometry}
\usepackage{fontspec}
\usepackage{unicode-math}
\usepackage{titlesec}
\usepackage{xspace}

\titlespacing*{\paragraph}{0pt}{3.25ex plus 1ex minus .2ex}{1em}
% \titlespacing*{\paragraph}{0pt}{1.0ex plus 1ex minus .2ex}{1em}
\setmainfont{Times New Roman}
\setmathfont{Cambria Math}
\pagenumbering{gobble}

\newcommand{\argmax}{\mathop{\mathrm{argmax}}}
\newcommand\lila{\textsc{L\={\i}la}\xspace}

\begin{document}

\paragraph{Motivation}

I first took an interest in linguistics 
while learning to speak Tagalog 
over the course of two years living in the Philippines. 
Up to that point I had never given much thought to language. 
I had scraped by in my high school Spanish class, 
but I only began to appreciate the complex relationship of language 
with culture, politics, and personal meaning 
once I had the chance to fully immerse myself in a language 
so distant from my native tongue.
% To study language is to catch a glimpse of something fundamentally human.
% To try to understand the computational,
% statistical, and information-theoretic foundations of language
% is an incredibly ambitious endeavor, 
% and one that I find deeply fulfilling.
My enthusiasm for language has not diminished since that time
and I am excited to pursue a PhD in natural language processing (NLP).

My educational and career goals are driven by three main factors.
The first is my passion for understanding language and reasoning
through developing computational models thereof.
This passion stems from my own curiosity,
as well as from my excitement for 
the large potential impact of NLP on society.
NLP is already beginning to transform us,
from enabling communication, 
to unlocking previously inaccessible insights from unstructured text. 
In particular, I am interested in studying language models
as general-purpose reasoning systems and better understanding 
what they can learn.

Secondly, I am driven by my enthusiasm for mentorship and education.
I have benefited from many amazing mentors, 
and I have had the honor of being a mentor and teacher to others, 
both in academic settings as a Harvard Teaching Fellow,
and outside of academia as a leader in multiple organizations 
committed to making the outdoors more accessible.
Good mentorship is key to the success 
and well-being of the next generation of scientists and thinkers. 
Furthermore, it can be an important avenue 
for those with under-privileged backgrounds to find their footing.

Lastly, I am driven by a sense of urgency for developing fair and safe AI tools.
AI has great potential for good in our world, 
but we are already seeing that progress comes with significant risks,
such as biased and toxic text generation.
Fairness and ethics in AI are perhaps the most pressing issues 
we face as progress in NLP accelerates.

\paragraph{Development plans}

My immediate education goal is to obtain a PhD 
where I can develop my research vision;
hone my mentorship, and teaching skills;
and receive the guidance I need 
to fulfill my eventual goal of becoming a principal investigator (PI). 
I hope to become a PI at an institution where I can continue to do impactful research
and continue to serve as a mentor to others.

\paragraph{Research experience}

My past research experiences have prepared me for graduate study
by giving me significant experience in ownership, development, and execution
of impactful research ideas. 
I have also gained important experience working with other researchers in both
large and small-group research settings. 
In this section, I detail some of my research experiences thus far.

Though some subfields of NLP research 
have advanced by leaps and bounds in the past few years,
our understanding of how these advances 
have come about remains unsatisfactory.
While transformers may get impressive results on, e.g., question answering,
we do not know what the fundamental limits 
of what current architectures can compute or approximate,
or what these models are doing internally.
Do they learn syntactic rules? 
Can they generalize to new problems composed of known concepts?
As I began my research career, I made the goal of answering these questions.

As my first step towards this goal, 
I set out to \textbf{interpret how modern language models handle syntactic agreement}
while an undergrad at Harvard, advised by Stuart Shieber and Yonatan Belinkov.
As a first author on this project, 
\textbf{I designed and ran compute-heavy experiments on a GPU cluster}
to intervene on individual neurons in large transformers 
and observe the resulting effects.
After running the experiments and obtaining results,
we discovered that another group 
(Aaron Mueller advised by Sebastian Gehrmann and Tal Linzen)
had independently ran very similar experiments and obtained almost identical results.
In our co-authored ACL 2021 paper~\cite{Finlayson2021CausalAO} 
we apply causal analysis to understand 
how transformers handle number agreement between nouns and verbs. 
We find that transformers generally learn two distinct mechanisms
for agreement, depending on whether the relevant tokens are adjacent or not.
Subsequent work has used causal analysis 
for analyzing other linguistic phenomena 
such as distributivity~\cite{Ban2022TestingPL},
and the effect of relative clauses on agreement~\cite{Ravfogel2021CounterfactualIR}.
Strengthening our understanding of how 
(and whether) 
language models deal with well-studied linguistic phenomena
can give insight into what inductive biases these models learn, 
and can guide research towards methods for improvement.

Excited by our findings, and starting as a pre-doctoral researcher at the Allen Institute for AI (AI2),
I decided to next explore how another field, formal language theory, 
can increase our understanding of the capabilities of transformer models.
In particular, I found formal language to be an excellent test-bed for
studying how well language models 
\textbf{generalize on highly compositional tasks}. 
Compositional generalization is one area in which 
neural methods have been shown 
to consistently underperform humans~\cite{Lake2018GeneralizationWS}.
This becomes particularly problematic 
as \textbf{instruction following} emerges 
as a prominent paradigm~\cite{mishra2021crosstask, Wei2021FinetunedLM}
for building general-purpose large language models. 
Instruction following is where a language model is expected to perform a novel task
(one not seen in training) given only a description of the task.
Since this paradigm has emerged relatively recently
little is known about what kinds of tasks current models can be expected to learn.
To evaluate this and provide a synthetic sandbox for future research,
I introduce RegSet~\cite{Finlayson2022WhatMI} in my \textbf{first-author EMNLP 2022 paper} 
advised by Kyle Richardson, Ashish Sabharwal and Peter Clark. 
In our work we propose a highly controllable proxy for studying instruction learning
by studying a language models ability to learn to interpret regular expressions.
Regular expressions are succinct representations of regular languages, 
a well studied and highly compositional class of formal languages.
As a result of our experiments,
we develop a handful of intriguing hypotheses 
about what makes instruction learning hard, 
including evidence that even large transformers struggle with modular counting 
(e.g., determining whether something is even or odd), 
less precise instructions, and tracking long contexts. 
We also release our challenging dataset, RegSet, to the public for further research.
Subsequent work~\cite{Merrill2022LogPrecisionTA} 
has built on our ideas by using formal language theory and computation theory
to characterize transformers ability to learn from instructions.
I am currently working with Ashish Sabharwal to further expand 
this framework and 
\textbf{find stronger theoretical bounds on what transformers can learn from inputs}.
My hope is that a better theoretical understanding 
of the limitations of transformers will inform 
our understanding of the architectural requirements 
for modeling language.

Building off of my work on learning from instructions, I am interested in
\textbf{learning how to leverage and control language models beyond fine-tuning}
by using techniques like in-context learning~\cite{Min2022RethinkingTR}.
In our preprint~\cite{Khot2022DecomposedPA},
working closely with Tushar Khot and advised by Ashish Sabharwal,
we develop methods for using large language models 
as problem decomposers and modular sub-problem solvers to improve performance over 
other prompting techniques such as chain-of-thought~\cite{Wei2022ChainOT}. 
As a future direction, I am interested in
developing a decoding method for using a language model 
to self-generate optimal task-specific prompts 
relying only on inference-time sequence probability estimates from the model.
Additionally, I am interested in developing methods 
for inference-time adversarial attacks on large language models. 
This line of work has the potential to uncover both highly efficient 
techniques for leveraging large language models,
and also uncover vulnerabilities that urgently need to be addressed 
in order to develop safe and fair AI systems.


While uncovering shortcomings of neural networks as general instruction followers with the RegSet work,
I became interested in also evaluating neural networks as 
\textbf{general-purpose math reasoners}.
In a contrasting approach,
I led the effort to compile a
\textbf{comprehensive natural language math reasoning benchmark} 
to evaluate language models' abilities on diverse 
math problem types.
I introduce the benchmark, \lila~\cite{Mishra2022Lila}, 
in my EMNLP 2022 paper 
(advised by Ashwin Kalyan and co-first-authored with Swaroop Mishra).
Current evaluation schemes fail to comprehensively capture 
general-purpose math reasoning skills in language models
because they are far too narrow in scope. 
As a result, they often overestimate the ability 
of particular models optimized for a single type of math reasoning.
In our large scale effort we draw together a diverse set of mathematical tasks 
and unite them under a single benchmark of over 140K math problems.
We provide valuable annotations for mathematical reasoning via program synthesis, 
where the language model has access to a Python interpreter.
Our experiments show that \textbf{multi-task learning}, 
combined with augmenting the model with a Python interpreter
improves general-purpose math reasoning
and the resulting model is an effective starting point 
for downstream fine-tuning. 
At the same time, our benchmark shows that language models, 
in their current form, 
are woefully deficient when it comes to math reasoning.
\textbf{I led and contributed to all aspects of the \lila paper},
including collecting and annotating the datasets,
the experimental design, running the experiments, and writing the paper.
Accomplishing this meant \textbf{leading a team of 11 researchers}.
My hope is that our high quality, comprehensive evaluation  
will serve to unify evaluation
towards developing general-purpose math reasoning models.

\paragraph{Intellectual merit}

My research career thus far has yielded several significant results for the field.
My initial work with causal analysis revealed interesting results for understanding
how transformers handle syntactic agreement, forming a bridge between 
linguistics and modern NLP. 
My subsequent project on learning from instructions provided empirical evidence for
important theoretical result on what transformers cannot learn, 
as well as provided a valuable tool and framework for studying instruction learning, 
an emerging paradigm in NLP.
\lila provides a much needed comprehensive benchmark 
towards evaluating math reasoning models, 
a major improvement over previous, fragmented evaluations.
Additionally, by releasing the models we developed with \lila, 
we provide future researchers with a strong starting point
for further model development.

\paragraph{Broader impacts}

Beyond purely academic findings and technical technical improvements,
my research has several broader societal implications.

Often, media and irresponsible research groups 
are incentivised to over-claim 
when it comes to the capabilities of new AI systems.
This AI hype can be dangerous 
because it often leads to overzealous deployment of these systems
and overconfidence in them.
This in turn can lead to risks as these models may unexpectedly 
regurgitate unchecked biases and toxicity from their training data,
or fail when it comes to mission-critical decision making.
With both RegSet and \lila, our datasets point out major shortcomings 
in language models' reasoning and instruction-following abilities.
These findings are important because they help combat AI hype, 
and ground claims about how well these systems actually work.

On the flip side, \lila represents an important step towards building 
reliable math reasoning systems. 
Such systems have significant potential for good, 
serving to assist and augment human achievements.
For instance, a reliable AI assistant capable of complex mathematical reasoning 
could help a person struggling to understand and control their personal finances.

\paragraph{Conclusion}

I have spent the past few years pursuing my passion for NLP 
and computational linguistics by
developing my research skills as an undergraduate at Harvard
and pre-doctoral investigator at AI2.
I am excited to pursue a PhD as the next chapter in my career
and further develop my knowledge, skills, and leadership as a researcher. 

\paragraph{References}[1] Ban, P., et al. Testing pre-trained language models’ understanding of distributivity via causal mediation
[2] Finlayson, M., et al. What makes instruction learning hard? an investigation and a new challenge in a
[3] Khot, T., et al. Decomposed prompting: A modular approach for solving complex tasks. ArXiv,
[4] Lake, B. M. and Baroni, M. Generalization without systematicity: On the compositional skills of
[5] Merrill, W. and Sabharwal, A. Log-precision transformers are constant-depth uniform threshold circuits.
[6] Mishra, S., et al. Cross-task generalization via natural language crowdsourcing instructions. In ACL.
[7] Wei, J., et al. Chain of thought prompting elicits reasoning in large language models.
[8] Wei, J., et al. Finetuned language models are zero-shot learners. In ICLR. 2022.


\clearpage
\bibliographystyle{nsf}
\bibliography{ref.bib}

\end{document}
